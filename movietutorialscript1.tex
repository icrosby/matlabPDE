\documentclass%[12pt] % uncomment the first % sign to enlarge the text for proofreading
{amsart}
\renewcommand{\baselinestretch}{1.5} % enable to get double spacing
\usepackage{verbatim, graphicx}

\theoremstyle{definition}
\newtheorem{example}{Example}

\begin{document}

\begin{LARGE}

What is a PDE?

To begin let's start with a definition of Ordinary Differential Equations.  In mathematics, an ordinary differential equation (or ODE) is a relation that contains functions of only one independent variable, and one or more of their derivatives with respect to that variable. A simple example of an ODE would be 

\[
f(x)''=-f(x)
\]

Where the solution to this would be 
\[
 f(x)=sin(x)+cos(x)
\] 
or any linear combination of $\sin(x)$ or $\cos(x)$

Another example of this would be newton's second law
\[
F=ma
\]
since acceleration is the second derivitive of the position equation.

Now on to PDE's.  A PDE is similar to an ODE except for PDE's deal with muti-variable function (ie $f(x,y)$

An example of an PDE would be the wave equation which models the motion of waves, such as in a vibrating string.  The wave PDE is

\[
u(t,x)_{tt}=c^2u(t,x)_{xx}
\]

What this says is that u is a function of t and x, and that the second derivitave of u with respect to t is equal to a constant squared times the second deravitive of u with respect to x.  One method of solving the wave equation deals with a technique called seperation of varables, and the solution is highly dependent on the initail conditions as well as the boundary conditions placed on the function.  

This eccentaily what a PDE is.  Now to Taha with examples of how we use PDE's.

\end{LARGE}

\end{document}